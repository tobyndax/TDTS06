\documentclass[10pt]{article}

\usepackage{times}
\usepackage{mathptmx}
\usepackage{amsmath}
\usepackage{mathtools}
\usepackage{graphicx}
\usepackage{epstopdf}
\usepackage{enumitem}
\usepackage[margin=1in]{geometry}

\setlength\parindent{0pt}


\raggedbottom
\sloppy

\title{Distance Vector Routing\\
\emph{TDTS06}}

\author{David Habrman \\ davha227, 920908-2412\\
Jens Edhammer \\ jened502, 920128-5112 }

\date{\today}

\begin{document}

\maketitle

\newpage
\tableofcontents
\newpage

\section{Distance Vector Routing}
Each router has a Distance Vector, which initially is set to the link costs to
it's neighbors and infinity for all others (since we know of no path to them).
In addition to the routers own Distance Vector each router has a DistanceTable
which can hold each neighbors Distance Vector, this is initialized as infinity
except for our own entry which is either set to our own Distance Vector or left
empty. (in our case we set it). After initializing everything the router sends
it's Distance Vector to each of it's neighbors, which means it's neighbors does
the same and we are now getting Distance Vectors in to our router. The first
thing we do is save each neighbors Distance Vector in the Distance Table. Then
we examine if there is a cheaper path to each destination based on the
information we have in the Distance Table. Note that you add the link cost 
towards the value in the Distance Table for comparison. If we do any changes
to our Distance Vector, either route choice or cost we notify all neighbors.
Repeat until convergence.

\section{Testing}
We tested our algorithm with and without poisoned reverse. We tested it on all
three cases with "LINKCHANGES" both on and off. Our final results was compared
with the expected values. The expected values were calculated by hand.

\section{Poison reverse fails}
Poison reverse fails if a link breakes a solution to the link break problem is
split horizon. This prevents loops by not sending a route back to the router
from which it was learned. For example, we have the routers A, B and C. A
connects to B and B connects to C (A and C only connect through B, A-B-C).
Split horizon prevents node A to send its route to C to router B since the
route goes through B. If we only have poison reverse there will be a loop
if the link between B and C breaks. B would learn a path to C through A,
then A would learn of a path to C through B and so on.

\section{References}
IP Routing Protocols: RIP, OSPF, BGP, PNNI, and Cisco Routing Protocols by  AvUyless D. Black \\
Computer Networking: A Top-Down Approach (6th Edition) by James F. Kurose and Keith W. Ross
\end{document}
