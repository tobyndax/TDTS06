\documentclass[10pt]{article}

\usepackage{times}
\usepackage{mathptmx}
\usepackage{amsmath}
\usepackage{mathtools}
\usepackage{graphicx}
\usepackage{epstopdf}
\usepackage[margin=1in]{geometry}

\setlength\parindent{0pt}


\raggedbottom
\sloppy

\title{TCP\\
\emph{TDTS06}}

\author{David Habrman \\ davha227, 920908-2412\\
Jens Edhammer \\ jened502, 920128-5112 }

\date{\today}

\begin{document}

\maketitle

\newpage
\tableofcontents
\newpage

\section{Introduction}

\section{Practice}
\begin{enumerate}
  \item{What are the first and last packets for the POST request?}
  \subitem{232129013 is first and 232293053 is last. First is a POST request,
    last is a reconstructed POST}
  \item{What is the IP address and the TCP port used by the client
    computer (source) that is transferring the file to gaia.cs.umass.edu?}
  \subitem{IP: 192.168.1.102 Port:1161}
  \item{What is the IP address of gaia.cs.umass.edu? On what port number
    is it sending and receiving TCP segments for this connections?}
  \subitem{IP: 128.119.245.12 Port:80}
  \item{What is the sequence number of the TCP SYN segment that is used
    to initiate the TCP connection between the client computer and
    gaia.cs.umass.edu? What is it in the segment that identifies the
    segment as a SYN segment?}
  \subitem{232129012, it has a header flag (SYN)}
  \item{What is the sequence number of the SYNACK segment
    sent by gaia.cs.umass.edu to the client computer in reply
    to the SYN? What is the value of the ACKnowledgement field
    in the SYNACK segment? How did gaia.cs.umass.edu determine
    that value? What is it in the segment that identifies the segment
    as a SYNACK segment?}
  \subitem{SYNACKseqNr: 883061785 ACK:232129013, SYNseqNr+1=ACK. Header flags
    (SYN and ACK). ACK:s everything up to but not including ACK number}
  \item{What is the sequence number of the TCP segment
    containing the HTTP POST command?}
  \subitem{232129013}
  \item{Consider the TCP segment containing the HTTP POST as the first segment
  in the TCP connection. What are the sequence numbers of the first six
  segments in the TCP connection (including the segment containing the
  HTTP POST)? At what time was each segment sent? When was the ACK for each
  segment received? Given the difference between when each TCP segment was
  sent, and when its acknowledgement was received, what is the RTT value for
  each of the six segments? What is the EstimatedRTT value (see page 277 in
  text) after the receipt of each ACK? Assume that the value of the
  EstimatedRTT is equal to the measured RTT for the first segment, and then
  is computed using the EstimatedRTT equation on page 277 for all subsequent
  segments.}
  \subitem{232129013,232129578,232131038,232132498,232133958,232135418.
    0.026477, 0.041737, 0.054026, 0.054690, 0.077405, 0.078157.
    0.053937, 0.077294, 0.124085, 0.169118, 0.217299, 0.267802.
    0.02746, 0.035557, 0.070059, 0.114428, 0.139894, 0.189645.
    0.02746, 0.028472, 0.033670, 0.043765, 0.055781, 0.072514.}
  \item{What is the length of each of the first six TCP segments?}
  \subitem{first is 565 bytes, all others are 1460 bytes}
  \item{What is the minimum amount of available buffer space advertised at
    the receiver for the entire trace? Does the lack of receiver buffer space
    ever throttle the sender?}
  \subitem{5840 form the SYNACK. It does not throttle the sender, we send much
    smaller packets (1460).}
  \item{Are there any retransmitted segments in the trace file? What did you
    check for (in the trace) in order to answer this question?}
  \subitem{No, we checked for tcp segments with the same sequence number as
    an earlier packet.}
  \item{How much data does the receiver typically acknowledge in an ACK?
    Can you identify cases where the receiver is ACKing every other
    received segment (see Table 3.2 on page 285 in the text).}
  \subitem{Usually it ACK:s only one segment (1460 bytes) but later on
    it ACK:s every other segment e.g. packet 105.}
  \item{What is the throughput (bytes transferred per unit time)
    for the TCP connection? Explain how you calculated this value.}
  \subitem{Total data: 150 KB, and the last ACK comes in at 5.651141 including
    the handshake. This gives a throughput of 150*8/5.651141 = 212.3 KBit/s.}
\end{enumerate}




\section{Questions}
\begin{enumerate}
  \item{What are the first and last packets for the POST request?}
  \subitem{}
\end{enumerate}


\section{Discussion}

\end{document}
