\documentclass[10pt]{article}

\usepackage{times}
\usepackage{mathptmx}
\usepackage{amsmath}
\usepackage{mathtools}
\usepackage{graphicx}
\usepackage{epstopdf}
\usepackage[margin=1in]{geometry}

\setlength\parindent{0pt}


\raggedbottom
\sloppy

\title{NetNinny\\
\emph{TDTS06}}

\author{David Habrman \\ davha227, 920908-2412\\
Jens Edhammer \\ jened502, 920128-5112 }

\date{\today}

\begin{document}

\maketitle

\newpage
\tableofcontents
\newpage

\section{Introduction}
This is a lab report in computer networks (TDTS06). The assignment was
to create a Web proxy using c++. The report will describe how to compile,
configure and use our Web proxy. It will also describe testing of the proxy,
limitations and finally a discussion about how the proxy handles different
types of websites.

\section{Manual}
The code is compiled using the makefile in appendix A. This makefile works on
ubuntu and OSX. Open the terminal, navigate to the folder containing code and
makefile, type "make", run the proxy with "./NetNinny xxxx". Replace "xxxx"
with the wanted the wanted socket number e.g. "5005" (required feature 7).
We preferred to use firefox to quickly test the proxy since it's easy to
modify the settings to use the proxy. Open firefox, open preferences, go to
the advanced tab, Network, Connection settings, check
"Manual proxy configuration", enter "local host" in the field "HTTP Proxy",
enter your socket number "xxxx" in the field "Port" and finally click "OK".

The code consists of three classes (see code in appendix B), proxy, comm and
parser. The proxy class handles the communication between the Web browser and
the proxy. The proxy class establishes a connection between browser and proxy,
then it waits for a request from the browser, this is then passed on to the comm
class. The comm class uses the parser class to check if the request contains a
undesirable URL (required feature 3). The comm class then uses the parser class
to modifying the request to accept no encoding and then send the request only if
the request passes the check, if not then it redirects to the specified error URL
("http://www.ida.liu.se/$\sim$TDTS04/labs/2011/ass2/error1.html"). The reason
for modifying the request to accept no encoding is for us to later be able to
scan content for inappropriate bytes.
It is the comm class that receives the messages from the server and before
it passes the message on to the proxy class it checks the message for any
text content (required feature 8). If the message contains any text then the
text content is passed through a check in the parser class to check for
inappropriate bytes (required feature 4). The message is passed on to the
proxy class if the content passes the check and the second specified error URL
("http://www.ida.liu.se/$\sim$TDTS04/labs/2011/ass2/error2.html") is returned if it
fails. All this together handles simple HTTP GET interactions between client
and server (required feature 2).

The message from the browser to the proxy ends with
"\textbackslash r\textbackslash n". We use this to read from the socket until
we find this string. The incoming message from the server is instead handled by
using "close connection" instead of "keep alive". This gives no limit on the
size of transferred HTTP data (required feature 5).

Our proxy blocks URLs and text content containing "norrk\"oping", "norrkoping",
"spongebob", "britney", "spears", "paris" and "hilton". We transform all content and
URLs to lower case letters which makes our proxy block insensitive to different
case letters.

\section{Testing}
We could categorize all websites into three different categories since the
proxy should handle three different cases, fully load a website,
block undesirable URLs, and block inappropriate content. We tested with some
different websites, among them we primarily used "www.aftonbladet.se"
(it was blocked due to inappropriate text content),
"britney.com" (it was blocked due to inappropriate URL) and "www.test.se" (it was passed through since
it was an appropriate website). Our proxy works for all cases and all websites.

We also tested streaming videos from "www.twitch.tv" which worked well.

\section{Features and Limitations}
Our proxy can handle HTTP GET interactions between client and server,
block requests for undesirable URLs and redirect them to an error URL,
modify requests to accept no encoding, detect inappropriate bytes in text
content and redirect to a second error URL, it allows the user to select the
proxy port and it has no limits on the size of the transferred HTTP data.

Since different browsers and servers are inconsistent with upper and lower case 
in the HTTP-headers we implemented some countermeasures against just that. 
We implemented our own comparator for the map-structure we use to extract and
 keep track of the different information from the HTTP-header, which was invariant 
 to upper and lower case letters. 
 
When looking for the Accept-Encoding: *****, to strip that line, we look for both  
accept-encoding and Accept-Encoding and remove the line if either is found.



\section{Discussion}
Our proxy works well on most websites, we can however not handle HTTPS.
Regarding gzip, we do modify the request to accept no encoding and thereby
search all content without having to unzip it. As mentioned above we can
handle both streaming and non-streaming websites.

\section*{Appendix A}
See makefile.

\section*{Appendix B}
See proxy.h, proxy.cpp, comm.h, comm.cpp, parser.h, parser.cpp and main.cpp.

\end{document}
